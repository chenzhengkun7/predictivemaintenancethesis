\chapter{Experiments}
\label{chapter:experiments}

In this chapter we describe experiments conducted on real-world publicly available \gls{pdm} data sets.
There are two main goals of the experiments:
\begin{itemize}
    \item demonstrate the approaches to \gls{pdm} utilizing \gls{ml} techniques;
    \item compare the different evaluation metrics that can be used within each approach.
\end{itemize}
For each of the approaches we conduct one experiment --- totaling in three experiments.

Though there is not enough publicly available data sets to make a robust comparison of the metrics, we can conduct each experiment on one carefully selected data set, suitable for the given approach, and compare the metrics from two aspects:
\begin{itemize}
    \item model selection --- Do evaluation metrics differ in how they rank different models for the task related to the data set?
    \item interpretability and practical value --- How can the evaluation metrics be interpreted by a domain expert? Does the evaluation bring practical value for the task related to the data set? For example do the metrics clearly express what the probability of predicting a failure or detecting a fault is?
\end{itemize}

As the approaches and the evaluation metrics are different, every experiment has slightly different experiment design.
The general design of the experiments can be, however, summarized as follows:
\begin{enumerate}
    \item candidate models selection --- use multiple evaluation metrics to select a set of candidate models from a larger amount of built models, e.g. select best ranked model by every metric, and discuss how the models agree in models' ranking;
    \item candidate models comparison --- compare the candidate models, e.g. using PR analysis, and discuss which model might be the most suitable model for the given task;
    \item performance interpretability --- discuss what will the model's performance in practice be.
\end{enumerate}

This chapter is organized as follows.
In Section \ref{sec:experiments_implementation} we describe implementation of the experiments including chosen technologies and the experiments' reproducibility.
In Sections \ref{sec:experiments_fault_detection}, \ref{sec:experiments_failure_prediction} and \ref{sec:experiments_rul} we describe the three conducted experiments, respectively.
Each experiment consists of sections:
\begin{itemize}
    \item data set description,
    \item task definition,
    \item design of experiment,
    \item results and
    \item discussion.
\end{itemize}

\section{Implementation}
\label{sec:experiments_implementation}

The full implementation of the experiments is available as a GitHub repository\footnote{\url{https://github.com/datamole-ai/predictivemaintenancethesis}}.
In this section, we provide a brief overview about the implementation details.

\subsection{Technologies}

We implemented the experiments as Jupyter notebooks \cite{jupyter} in Python using standard Python libraries for machine learning, scientific computations, and visualizations including Scikit-learn \cite{scikit-learn}, NumPy \cite{numpy}, SciPy \cite{2020SciPy-NMeth}, Matplotlib \cite{matplotlib}, Seaborn \cite{seaborn}, XGBoost \cite{xgboost} and Pandas \cite{pandas}.
For the calculation of range-based precision and recall metrics we used a python implementation the authors of \cite{tatbul2018precision} provide at GitHub \cite{tsmetric}.
Moreover, for the calculation of \acrshort{auprg} we used pyprg package \cite{prg}.

\subsection{Hardware}

We ran the experiments on a computational cluster using 256 GB of RAM and 32 CPU cores of AMD Opteron 6344 CPU units provided by Datamole\footnote{\url{www.datamole.cz}}.

% \subsection{Implementation Details}

% TODO

% Scikit-learn \cite{scikit-learn} is one of the most commonly used Python libraries for machine learning which contains many standard classification and regression algorithms and TODO.
% In \gls{pdm} data sets we often operate with the data for each subject separately, e.g. artificial labeling in failure prediction or calculating e.g. rolling features should be done for each subject separately.
% Therefore, we implement a wrapper class over a scikit-learn's 

% All the evaluation metrics are available as standalone functions compatible with 
% All the experiments are are provided as a GitHub repository (TODO: cite).
% The implementation includes downloading the data from the web,
% preprocessing and data transformation, and all the evaluation metrics described
%  in the previous chapter.
% The experiments thus are reproducible and extensible, and their parts
% are reusable in other works.\markbelow{c}{
%     This section might be extended with the description of the Python library
%     - if it will be part of the Work.
% }

% % Each experiment also contains script for downloading the data 

% The parts of the experiments where algorithms dependent on a random seed (e.g. a random search algorithm for optimizing hyperparameters) use a hardcoded value of the seed.

% The experiments are completely reproducible.

% sklearn model for data in series format
% implementation of various metrics and series_scorer wrapper


\subsection{Reproducibility}

Each experiment is completely reproducible.
We use Poetry package manager \cite{poetry} to ensure that the same versions of Python libraries we used can be installed for reproducing the experiments.
Our GitHub repository thus contains a \texttt{pyproject.toml} and \texttt{poetry.lock} files which can be used to install all the Python packages at the same version we used.
We used a fixed random seed in all parts of code that depend on randomness so that reproducing the experiments always yields the same results.
Reproducing each experiment then consist in running a Jupyter notebook which contain code for downloading the publicly available data set, preprocessing of the data and all the other steps of the experiments such as modeling and visualizing the results.
\section{Experiment --- Fault Detection in Scania Trucks}
\label{sec:experiments_fault_detection}

Fault detection is an approach where the goal is to build a model that detects faulty behaviour, malfunction, of the subject.
It can be modeled as a binary classification or an anomaly detection --- depending on whether health labels are available or not.
It is suitable in cases when there are no or insufficient data about actual failures, i.e. breakdowns, of the subjects.
In this section we describe an experiment where we demonstrate this approach and compare its evaluation metrics on one real-world data set.

The data used for building a fault detection model can contain either point-based or range-based faults, i.e. faults with no temporal location or faults located in time and lasting for a certain period of time (for more details see Section \ref{sec:approaches_fault_detection_data}).
Since detection of range-based faults is partly similar to failure prediction approach, which we demonstrate in the following experiment, we focus in this experiment on fault detection of point-based faults.

For the purpose of our experiment we choose a data set containing point-based faults in air pressurized system of Scania trucks \cite{Dua:2019}.
We choose this data set because its authors clearly define an objective function --- a cost function assigning costs to false alarms and missed faults (FPs and FNs) expressed in the amount of dollars.
The authors then clearly define the task as to build a model that minimize this cost function.
The cost function allows us to demonstrate how decision threshold of the built classifier can be selected in practice.

\subsection{Data Set Description}

The data set we use in this experiment contains condition monitoring data and point-based faults in an air pressure system of heavy Scania trucks \cite{Dua:2019}.
It consists of 76000 records and 171 columns where each row represents one truck with the first column containing a binary health label (positive class represents a fault) and the next 170 columns containing anonymized features.
The provided data are  already split into training and testing set where the training set contains 60000 records and the testing set contains 16000 records.
Both the sets are highly imbalanced --- the ratio of positive classes to the total amount of records is approximately 1.67 \% in the training set and 2.34 \% in the testing set.

\paragraph{Preprocessing}

The data set is provided partially preprocessed in a form of two CSV files, one for training set and one for testing set.
All the features are numerical of which some of them are provided already binned --- i.e. split to a finite number of intervals. 
There are some missing values in multiple feature columns which we impute by a mean of each column in the training data\footnote{Note that it is very important not to impute missing values by a mean values of the whole data set as that would contaminate the data training data with the information from the testing data.}.
Otherwise, we consider the data set as preprocessed and suitable for the classification algorithm we chose (described later below).

\subsection{Task Definition}

The authors of the data set define a cost function assigning costs in dollars for the false predictions:
\begin{itemize}
    \item \$ 10 per FP --- cost of unnecessary check needed to be done by a mechanic at workshop;
    \item \$ 500 per FN --- missing a faulty truck that may cause a breakdown.
\end{itemize}
The task the authors set is to thus build a fault detection model using the training set that predicts whether a fault is present in the truck and to minimize total cost of the false predictions in the testing set.

\subsection{Design of Experiment}

\begin{figure}
	\centering
    \includegraphics[width=\textwidth,keepaspectratio]{%
        experiments_fault_detection_design.pdf}
	\caption{Fault detection in Scania Trucks: Design of experiment.}
	\label{fig:experiments_fault_detection_design}
\end{figure}

Our data set contains enough labels for both positive and negative samples and thus we can approach the fault detection as a supervised classification.
For the purpose of our experiment, we choose only one type of classification algorithm --- gradient boosted trees, more specifically its Python implementation XGBoost \cite{xgboost}.
XGBoost has several hyperparameters that should be tuned such as number of trees (estimators), max depth of the trees or minimal number of samples to perform a further split (min\_child\_weight).
Moreover, the XGBoost is a probabilistic classifier, i.e. it can predict probabilities instead of the binary classes themselves.
A decision threshold that optimizes minimizes the total costs can be thus tuned.

We train and evaluate a large amount of XGBoost models with different hyperparameters on a subset of the training data using various metrics and a cross-validation.
We compare how the metrics rank the different models and we select a set of candidate models --- the models that are ranked as best by at least one metric.
Afterwards we compare the candidate models in terms of precision and recall over various thresholds on rest of the training set --- a validation set.
Since we have a clearly defined cost function, we select as the best model (out of the candidate models) the one that achieves minimal total cost and we set it the corresponding decision threshold.
Once we select the final model we evaluate it using the testing set and discuss its real-world performance.

The individual steps of the experiment are described in details below and illustrated in Figure \ref{fig:experiments_fault_detection_design}.

\subsubsection{Data Splitting}

The data set is provided already split into a training and testing set, as described above.
For our experiment we need one more set --- a validation set --- which we will use for selecting the best decision threshold.
Therefore we split the original training set into new training set and a validation set with a ratio 4:1 (i.e. the new training set thus contains 48000 records and the validation set 12000 records).
We perform the split in a stratified way so that the ratio of positive and negative samples remains the same in the new training and the validation sets.
When speaking about a training set we will from now on refer to this new training set.

\subsubsection{Candidate Models Selection}

\begin{table}
    \centering
    \begin{tabular}{cc}
    hyperparameter
    & values\\
    \hline
    max\_depth & ${2, 3, 4, 5, 6, 7}$\\
    n\_estimators & ${4, 8, 16, 32, 64, 128}$\\
    learning\_rate & ${0.05, 0.1, 0.15, 0.2}$\\
    booster & {gbtree, dart}\\
    min\_child\_weight & ${ 1,  4, 16, 64}$\\
    subsample & ${0.6, 0.7, 0.8, 0.9, 1}$\\
    colsample\_bytree & ${0.6, 0.7, 0.8, 0.9, 1}$\\
    \end{tabular}
    \caption{Fault detection in Scania trucks: Set of tuned hyperparameters for the XGBoost}
    \label{tab:experiments_fault_detection_aps_hyperparameters}
\end{table}

We use the training set and a random search algorithm with cross-validation to train and evaluate multiple XGBoost models with different hyperparameters.
Table \ref{tab:experiments_fault_detection_aps_hyperparameters} shows the set of hyperparameters we select from.
For evaluation we use 10-fold cross-validation and we evaluate each model with four metrics --- AUROC, AUPRG, F1 score and accuracy --- and we calculate a mean score of the respective metrics over the testing folds.
F1 score and accuracy are both calculated on the predictions made by using the decision threshold equal to 0.5.

Training and evaluation of one model takes approximately one minute --- the cross-validation (10 folds) for one set of hyperparameters thus takes approximately 10 minutes.
As we have available 32 CPU cores we run 160 iterations of the random search (5 for each CPU) so that it takes approximately 50 minutes.
As a result we obtain a list of 160 models each assigned four scores and four ranks (each rank in range $[1, 160]$ with rank 1 being the best).

To compare the rankings of the models by the evaluation metrics we visualize a pairplot of the models' ranks.

Finally, from the trained models we select a set of candidate models.
A candidate model is a model that is ranked as the best model by at least one of the evaluation metrics.

\subsubsection{Final Model and Decision Threshold Selection}

We retrain every candidate model using the whole training set and we use the validation set to calculate and visualize precision, recall and total cost (calculated as \$ 10 for FP and \$ 500 for FN) over various decision thresholds.
As the final model we select the model with the lowest total cost and we set it the corresponding decision threshold.

\subsubsection{Evaluation and Performance Interpretation}

We use the testing set to calculate accuracy, precision, recall and total cost.
We interpret the metrics and discuss how the model performs in real-world.

\subsection{Results}

\begin{figure}
	\centering
    \includegraphics[width=\textwidth,keepaspectratio]{%
        experiments_fault_detection_aps_pairplot.pdf}
	\caption{Fault detection in Scania Trucks: Pair plot of the evaluation metrics obtained from the random search.}
	\label{fig:experiments_fault_detection_aps_pairplot}
\end{figure}

Figure \ref{fig:experiments_fault_detection_aps_pairplot} shows a pair plot of rankings of all the XGBoost models.
We can see that F1 and accuracy agree on the ranking of the models as well as AUROC and AUPRG do.
However, the two pairs disagree with each other, i.e. both AUROC and AUPRG disagree with both F1 and accuracy.
This is not very surprising as the F1 and accuracy are based only on predictions at threshold 0.5 whereas AUROC and AUPRG evaluate the model over all the thresholds.

\begin{table}
    \centering
    \begin{tabular}{lll}
    \toprule
    {} & \multicolumn{2}{c}{candidate model} \\
    {} &         A &         B & \\
    \midrule
    rank by AUPRG               &         1 & 33 \\
    rank by F1                  &        23 & 1 \\
    rank by ACCURACY            &        23 & 1 \\
    rank by AUROC               &         1 & 49 \\
    AUPRG score                       &  0.999812 & 0.999767 \\
    F1 score                          &  0.808412 & 0.831349 \\
    ACCURACY score                    &  0.994146 & 0.994875 \\
    AUROC score                       &  0.990641 & 0.988216 \\
    XGBoost param:subsample        &       0.9 &        0.8 \\
    XGBoost param:n\_estimators     &       256 &   256 \\
    XGBoost param:min\_child\_weight &        16 &      1 \\
    XGBoost param:max\_depth        &         5 &     7 \\
    XGBoost param:learning\_rate    &       0.1 &       0.2 \\
    XGBoost param:colsample\_bytree &       0.7 &       0.9 \\
    XGBoost param:booster          &      dart & dart \\
    \bottomrule
    \end{tabular}
    \caption{Fault detection in Scania trucks: Ranks, scores and parameters of the candidate models}
    \label{tab:experiments_fault_detection_aps_selected_models}
\end{table}

We identify two candidate models, i.e. models that are ranked by at least one metric as the best model
Table \ref{tab:experiments_fault_detection_aps_selected_models} shows their ranks and evaluation scores and the XGBoost's hyperparameters.
As expected, one of the models, model A, is selected by AUROC and AUPRG while the other, model B, is selected by F1 and accuracy.
Based on the XGBoost's hyperparameters we can see that the model A has lower complexity than the model B.
The model A has the same number of estimators (trees) but it has lower maximum depth and higher minimal child weight\footnote{the minimal child weight defines the minimal number of samples in the node so that the node can be further split}.
The lower complexity models are always more preferable as they have a lower risk of being overfitted.
Therefore, from the current view we assume the model A as better so far.

% \subsubsection{Final Model and Decision Threshold Selection}

\begin{figure}
    \centering
    \begin{subfigure}{\textwidth}
        \includegraphics[width=\textwidth,keepaspectratio]{%
        experiments_fault_detection_aps_cost_threshold_auroc.pdf}
        \caption{Model selected by AUROC and AUPRG}
    \end{subfigure}
    \begin{subfigure}{\textwidth}
        \includegraphics[width=\textwidth,keepaspectratio]{%
        experiments_fault_detection_aps_cost_threshold_f1.pdf}
        \caption{Model selected by F1 and accuracy}
    \end{subfigure}
    \caption{Fault detection in Scania trucks: Precision-recall-cost plot for the candidate models.}
    \label{fig:experiments_fault_detection_aps_cost_threshold}
\end{figure}

Figure \ref{fig:experiments_fault_detection_aps_cost_threshold} shows plots of costs and precision and recall scores over various decision thresholds for both of the candidate models.
We can see that the optimal threshold, i.e. the threshold where the cost is minimal, is very low and thus the recall is significantly higher than the precision.
This is because the cost of the FNs is much higher than of the FPs and thus it is better to have as few FNs as possible, i.e. having a high recall.
The lowest costs per model (annotated in the figure) are as follows:
\begin{itemize}
    \item model A: \$ 5210 at decision threshold $2.4e^{-2}$;
    \item model B: \$ 7290 at decision threshold $1.3e^{-3}$.
\end{itemize}
The results thus confirm that the model having the lower complexity, the model A, selected by AUPRG and AUROC, is better.
Therefore, we select the model A as our final model and we set its decision threshold to $2.4e^{-2}$.

The evaluation of the final model (model A) on the testing set and using the decision threshold $2.4e^{-2}$ gives following results: 
\begin{itemize}
    \item Recall: 0.99
    \item Precision: 0.17
    \item Cost: 18950
\end{itemize}
The recall can be translated as that the model will detect 99 \% of faults.
The precision can be translated as that only 17 \% of positive predictions will actually correspond to a faulty truck, or in other words 83 \% of the predictions will be false alarms.

\subsection{Discussion}

In this experiment we demonstrated how to build a binary classification model for detection of point-based faults of Scania trucks.
The results of the experiment show that AUPRG and AUROC metrics were better choice for model selection than F1 and accuracy metrics this data set and the XGBoost model.
That is because there might be different importance of FP and FN (in our experiment we needed to have fewer amount of FN than FP) and thus it might be better to select a model that performs well at all thresholds and leave the decision threshold selection for later, when the specific domain needs are known.

Regarding evaluation of the model's performance in practice, we demonstrated that precision and recall can nicely interpret the model's performance, i.e. the probability that a fault will be detected and how often will the model predict a false alarm.
It is good to note though, that the metrics cannot take into account any time information, e.g. how early will be the faults detected, as the point-based data do not contain any time information.
To include the temporal information in evaluation, one has to use either data with range-based faults or another \acrshort{pdm} approach.
\section{Experiment --- Failure Prediction in Azure Telemetry Data Set}
\label{sec:experiments_failure_prediction}

This section describes an experiment where we demonstrate failure prediction approach and we compare its evaluation metrics.
Failure prediction is an approach where the goal is to build a model that predicts whether a failure will happen in near future --- in the monitoring window.
This approach is suitable in cases when there are available data about failures and when the failures are expected to be preceded by a faulty behaviour of the subject.

The modeling typically consist of formulating the problem as a binary classification where before training the classifier, the samples prior to the failure are artificially labeled as positive.
This, however, introduces challenge in the model's evaluation as there are more positive samples than failures.
In Section \ref{sec:approaches_failure_prediction_evaluation} we described how classical precision and recall metrics can be modified so that they provide more realistic scores.
We call the modified metric event-based precision and event-based recall.
In this experiment, our goal is to compare the classical and event-based metrics in terms of model selection, the model's decision threshold selection and interpretability.

As said, failure prediction consists in predicting whether a failure will happen in the monitoring window.
The size of the monitoring window can be either predefined by the domain (e.g. it might be known that the faulty behaviour of the subjects lasts no longer than 7 days before the failure) or it can be tuned as a hyperparameter.
For the purpose of this experiment, we choose a data set that has the size of the monitoring window already predefined by the domain experts.
The data set we chose is a publicly available Azure AI Gallery data set \cite{data_set_azure_ai_gallery} which contains multiple data sources like sensor measurements and failure logs about 100 machines and the authors of the data set clearly define the task: predict whether a failure will happen in next 24 hours.

\subsection{Data Set Description}

\begin{figure}
	\centering
    \includegraphics[width=\textwidth,keepaspectratio]{%
        experiments_failure_prediction_azure_data.png}
	\caption{Failure prediction in Azure data set: Example of one machine's data. The vertical dotted lines represent the failure events.}
	\label{fig:experiments_failure_prediction_azure_data}
\end{figure}

The data set we use in this experiment is an Azure AI Gallery Predictive Maintenance data set \cite{data_set_azure_ai_gallery} which contains continuously collected condition monitoring data and failure labels of an unspecified machinery.
The data consist of telemetry data, error logs, maintenance logs and failure logs for 100 machines collected during whole year of 2015.
The telemetry data include voltage, rotation, pressure and vibration measurements and are collected on an hourly basis --- one value per hour.
The error log contains time stamped information about non-breaking errors.
The maintenance log contain time stamped events of both scheduled maintenance actions (regular inspection) and unscheduled maintenance actions (failures) .
The failure logs contain time stamped information when the failures happened.
When a failure happens on the machinery the failure is repaired and the machinery is put to operation again.

\paragraph{Preprocessing}

The data are available as separate CSV files for telemetry data and error, maintenance and failure logs.
As the telemetry data are available on an hourly basis we round the time stamps of all the events, i.e. maintenance actions, errors and failures, to the closest hour and join the data on time stamps and machine identifications.
To help the classifier identify temporal patterns in the data we create following time-based features (for every time point): 
\begin{itemize}
    \item mean, variance and sum of the telemetry data for the past 7 days;
    \item time from the last maintenance action, from the last error and from the last failure.
\end{itemize}
Figure \ref{fig:experiments_failure_prediction_azure_data} shows an example of the telemetry data for one machine.

\subsection{Task Definition}

The task, defined by the authors of the data set, is to predict whether a failure will happen in next 24 hours.
The authors do not mention any warning window necessary for the predictions to be useful (e.g. so that there is enough time for the maintenance to be scheduled).
Though it might be that there is no warning window necessary, we assume that is highly unlikely in practice and we assume the authors of the data set probably have not thought about the possibility of defining a warning window.
Therefore, we set the warning window ourselves to 8 hours, i.e. one third of the monitoring window.
This means that during the evaluation, all the predictions made less than or equal to 8 hours prior to the failure will be ignored.
The size of our prediction window, i.e. the size of an interval prior to the failure where the training samples are considered as positive, is thus 16 (monitoring window minus warning window).

\subsection{Design of Experiment}

We approach the task as a supervised binary classification problem where we artificially label all the samples 24 hours (size of monitoring window) prior to each failure as positive to train the model.
Regarding evaluation of the model, we are mainly interested in precision and recall, i.e. a probability that a true prediction actually predicts the failure and a probability of predicting a failure.
However, as there are more true positive samples than the amount of failures it is not straightforward way how to use these metrics.
In Section \ref{fig:approaches_failure_prediction_evaluation} we described a concept of event-based precision and recall where the TPs are replaced by detection scores and FP are replaced by discounted FP.
Therefore, we design our experiment as to compare how these event-based metrics affect the model selection and decision threshold selection in comparison with the classical precision and recall.

The event-based metrics use detection score which has four parameters that can be set based on the domain specific needs.
The parameters are a weight between existence and overlap ($\alpha$), a cardinality function ($\gamma$), an overlap function ($\omega$) and a positional bias function ($\delta$).
For more details about the parameters see Section \ref{sec:approaches_fault_detection_evaluation}.
For our task, we set $\alpha = 0.8$ as we are rather interested in the existence of a true prediction in the prediction window than the amount of overlap.
The cardinality, i.e. whether the predictions are fragmented, does not matter in failure prediction that much and thus we set it to be always equal to one.
As an overlap function we use standard suggested definition in Section \ref{sec:approaches_fault_detection_evaluation} and we set the positional bias to flat --- i.e. we do not distinguish whether the prediction is at the beginning of the prediction window or at the end.
As result, the detection score of every even is thus either equal to 0 (when there are no predicted positive samples) or is in range $[0.8125, 1]$\footnote{0.8 for the existence of a positive prediction in the prediction window plus $0.2 times 1/16$ (0.0125) for every positive prediction}, depending on the amount of positive predictions in the prediction window.

As the classification algorithm we choose gradient boosted trees, more specifically XGBoost \cite{xgboost}, which is capable of predicting probabilities of classes and thus the decision threshold can be tuned.

Compared to classical binary classification, in failure prediction the predictions can be smoothed.
It can for example happen that the model will predict a lone positive prediction among negative predictions which can be caused for example by a noise in the data.
Therefore, the predictions are typically smoothed e.g. using a rolling mean where the predicted probability of each samples is calculated as the mean of several past predictions (including the current).
For more details about prediction smoothing see Section \ref{sec:approaches_failure_prediction}.
In this experiment, we use the rolling mean for smoothing of predicted probabilities and we tune the smoothing window size as a hyperparameter.

We design the experiment to consist of three steps (illustrated in Figure \ref{fig:experiments_failure_prediction_design}):
\begin{itemize}
    \item data splitting --- split the data into training and testing set;
    \item candidate models selection --- use the training set and cross-validation to train and evaluate XGBoost model with different hyperparameters and smoothing window sizes, compare how the metrics rank the trained models and select a set of candidate models, i.e. models that are ranked at least by one metric as best;
    \item PR analysis --- analyze candidate the models on the testing set using both classical and event-based precision and recall, discuss which model is the most suitable for the given task and compare the metrics' interpretability;
\end{itemize}
The individual steps of the experiment are in detail described below.

\begin{figure}[H]
    \centering
        \includegraphics[width=.8\textwidth]{%
            experiments_failure_prediction_design.pdf}
    \caption{Failure prediction in Azure data set: Design of experiment}
    \label{fig:experiments_failure_prediction_design}
\end{figure}

\subsubsection{Data Splitting}

We split the data into training and testing set.
We identified two plausible splitting strategies: split by time and split by subject.
The former consists in selecting the newest data (e.g. last two months, November and December) as the testing and the older data as the training.
However, such splitting strategy might make the model to learn some subject specific patterns.
Therefore, we adopt the latter splitting strategy: split by subject.
We split the data set as follows:
\begin{itemize}
    \item training set: 80 subjects
    \item testing set: 20 subjects
\end{itemize}

\subsubsection{Candidate Models Selection}

We use the training set to train and evaluate a large amount of XGBoost models with different XGBoost's hyperparameters and smoothing windows and we select a set of candidate models --- models ranked as best by at least one metric.

As the evaluation metrics we use AUPRG, F1 score and event-based F1 score.
The event-based F1 score is calculated based on event-based precision and recall metrics with the parameters as described above.
The AUPRG (area under precision-recall-gain curve) is calculated based on classical precision and recall.
We do not use the event-based metrics to calculate the area under event-based PR curve as the calculation of it is extremely computationally expensive.
The regular AUPRG is calculated by sorting the samples by predicted score and the amount of true positive and false positive predictions can be then calculated using a cumulative sum operations \cite{pr_efficient}.
Regarding the event-based metrics, however, the precision and recall have to be calculated separately for each threshold.
The calculation of e.g. hundreds of thresholds then can take tens of minutes which is more than the amount of time for training the model itself.
Therefore, we do not calculate the area under event-based precision-recall(-gain) curve and we use only the event-based F1 score calculated based on predictions made by the default decision threshold 0.5.

We run a random search algorithm with three-fold cross validation to train and evaluate the models.
The average training time of one fold is 10 minutes.
Since we have 32 CPU cores available we run 64 random search iterations so that the total computation time is approximately one hour.
The models are then assigned an average score over the testing folds and are assigned a corresponding rank for every metric.
Every trained model has thus assigned ranks per each metric in range $[1, 64]$ where rank 1 stands for the model with best score and rank 64 stands for the model with worst score.

\begin{table}
    \centering
    \begin{tabular}{ll}
    hyperparameter
    & values\\
    \hline
    smoothing window & $\{1, 3, 5, 7\}$ \\
    XGBoost: max\_depth & $\{2, 3, 4, 5, 6, 7\}$ \\
    XGBoost: n\_estimators & $\{4, 8, 16, 32, 64, 128, 256\}$ \\
    XGBoost: learning\_rate & $\{0.05, 0.1, 0.15, 0.2\}$ \\
    XGBoost: booster & \{'gbtree', 'dart'\} \\
    XGBoost: min\_child\_weight & $\{ 1,  4, 16, 64\}$ \\
    XGBoost: subsample & $\{0.6, 0.7, 0.8, 0.9, 1\}$ \\
    XGBoost: colsample\_bytree & $\{0.6, 0.7, 0.8, 0.9, \}$ \\
    \end{tabular}
    \caption{Failure prediction in Azure data set: Set of tuned parameters.}
    \label{tab:experiments_failure_prediction_azure_parameters}
\end{table}

The hyperparameters we optimize are hyperparameters of the XGBoost algorithm and a size of the smoothing window.
Table \ref{tab:experiments_failure_prediction_azure_parameters} summarizes all the tuned hyperparameters and the set of values we select from.

Once we have all the models evaluated we visualize a pairplot to compare ranks of the individual models for every pair of the four evaluation metrics.
Afterwards, we select a set of candidate models from the best ranked models, i.e. models ranked high by at least one metric.

\subsubsection{PR Analysis}

We use testing set to perform PR analysis of the candidate models selected in previous step.
For every candidate model we calculate both classical and event-based precision and recall over different thresholds and visualize them.
We then discuss whether and how the candidate models differ and whether the classical and event-based metrics differ in decision threshold selection.

\begin{figure}[H]
	\centering
    \includegraphics[width=\textwidth,keepaspectratio]{%
        experiments_failure_prediction_azure_correlation.pdf}
	\caption{Failure prediction in Azure data set: Rankings of the models by various metrics based on the mean metrics' values on the testing cross-validation folds.}
	\label{fig:experiments_failure_prediction_azure_correlation}
\end{figure}

\begin{table}[H]
    \centering
    \begin{tabular}{lllll}
    \toprule
    {} & \multicolumn{3}{c}{candidate model} \\
    {} & \multicolumn{1}{c}{A} &    \multicolumn{1}{c}{B} & \multicolumn{1}{c}{C}\\
    \midrule
    rank by AUPRG                     &         1 &        16 &        44 \\
    rank by classical F1                        &        24 &         9 &         1 \\
    rank by event-based F1                &         9 &         1 &        10 \\
    AUPRG score                             &  0.999997 &  0.999991 &  0.999954 \\
    classical F1 score                                &   0.95681 &  0.964553 &  0.969436 \\
    event-based F1 score                        &  0.955075 &  0.961267 &  0.953946 \\
    param\_smoothing\_window            &         1 &         7 &         1 \\
    param\_estimator\_\_subsample        &       0.7 &       0.7 &       0.9 \\
    param\_estimator\_\_n\_estimators     &        64 &        64 &       256 \\
    param\_estimator\_\_min\_child\_weight &        16 &        16 &         1 \\
    param\_estimator\_\_max\_depth        &         7 &         5 &         4 \\
    param\_estimator\_\_learning\_rate    &      0.15 &       0.2 &       0.2 \\
    param\_estimator\_\_colsample\_bytree &       0.6 &       0.6 &       0.9 \\
    param\_estimator\_\_booster          &      dart &      dart &      dart \\
    \bottomrule
    \end{tabular}
    \caption{Failure prediction in Azure data set: Ranks and parameters of the candidate models.}
    \label{tab:experiments_failure_candidate_models}
\end{table}

\subsection{Results}

Figure \ref{fig:experiments_failure_prediction_azure_correlation} shows a pair plot of the rankings of the 64 trained models.
We can see that F1 and event-based F1 relatively agree in the ranking though they select slightly different model as best.
AUPRG, on the other hand, highly disagrees with both F1 and event-based F1.
For example the best model selected by F1 has rank between 40 and 50 (with 64 being the worst) by AUPRG.

Table \ref{tab:experiments_failure_candidate_models} shows scores, ranks and hyperparameters of the models ranked as best by at least one metric.
We can see that the models chosen by F1 and AUPRG have both smoothing window of size 1 while the model chosen by event-based F1 has smoothing window of size 7.
This might be caused by the low amount of lone FP, i.e. the smoothing only unnecessarily delays the positive predictions and thus causes the precision and recall scores to be.
As both the classical and event-based F1 scores are very high --- above 95 \% --- we can assume that most of the failures were predicted and that the predictions are made in the most of the prediction windows.

\begin{figure}
    \centering
    \begin{subfigure}{.85\textwidth}
        \includegraphics[width=\textwidth]{%
            experiments_failure_prediction_azure_pr_curves_ts.pdf}
        \caption{Classical PR curves}
    \end{subfigure}
    \begin{subfigure}{.85\textwidth}
        \includegraphics[width=\textwidth]{%
            experiments_failure_prediction_azure_pr_curves_reduced.pdf}
        \caption{event-based PR curves}
    \end{subfigure}
    \caption{Failure prediction in Azure data set: Classical and event-based PR curves of the candidate models. Note, that both the x-axis and y-axis have range from 0.6 to 1.}
    \label{fig:experiments_failure_prediction_azure_pr}
\end{figure}

\begin{figure}
    \centering
        \includegraphics[width=\textwidth]{%
            experiments_failure_prediction_azure_multicurve_f1.pdf}
    \caption{Failure prediction in Azure data set: Classical and event-based precision, recall and F1 scores over decision thresholds for the model selected by F1 score (model C). Note, that the y-axis has range from 0.86 to 1.}
    \label{fig:experiments_failure_prediction_azure_multicurve}
\end{figure}

Figure \ref{fig:experiments_failure_prediction_azure_pr} shows both the classical and event-based PR curves for all the three candidate models.
We can see that the models selected by classical F1 score (green curve) and AUPRG (blue curve) are comparable in terms of classical PR curve.
However, the model selected by classical F1 has significantly better event-based PR curve than the model selected by AUPRG.
This is surprising as the model selected by F1 score was scored very low by the AUPRG metric (rank 44 out of 64).
It therefore suggests that a model that has good AUPRG score does not have to perform well regarding event-based metrics.
Regarding the model selected by event-based F1 score, we can clearly see that it has significantly worse classical PR curve than the other two models.
This is most probably caused by the smoothing --- the predictions at the beginning of the prediction window might have low probabilities.
Regarding event-based PR curve, however, we see that the model selected by event-based F1 score is slightly better at high event-based precision values than the model selected by classical F1 score.
This suggests that the smoothing might help achieve better results when high precision is important, i.e. when the false alarms are costly.

If precision would be of high importance, we would choose the model selected by the event-based F1 score that smooths the predictions.
However, since we do not know the exact costs of FP and FN, we choose the best performing model in overall.
That is the model selected by classical F1 score as it has superior both classical PR and event-based PR curves over the other two models.
Therefore, we use this model to compare how the classical and event-based differ in the decision threshold selection.

Figure \ref{fig:experiments_failure_prediction_azure_multicurve} shows classical and event-based precision, recall and F1 scores over various decision thresholds for the model selected by classical F1 score.
We can see that the event-based precision is significantly lower than the classical precision.
That is caused by using the detection score instead of true positives and by using the discounted FP instead of standard FP.
The size of the difference between the classical and event-based precision provides an insight into how are the FP close together --- the higher the difference the more distant are the FP from each other.
The event-based recall, on the other hand, is higher than the classical recall.
That is caused by using the detection score with an existence reward, i.e. having only a single prediction in the prediction window causes the detection score to be $> \alpha$, i.e. at least as big as the existence weight.
In our case $\alpha = 0.8$, as mentioned in the experiment design. 
This then leads to the highest value of event-based F1 score being at higher decision threshold (approx. 0.8) than the highest value of classical F1 score (approx. 0.4).
In other words, the event-based metrics suggests that predicting only the samples that the model is more confident about as positive, i.e. selecting higher decision threshold, is likely to bring better precision without much of a decrease in recall.

\subsection{Discussion}

In this experiment we demonstrated failure prediction approach where we formulated the problem as a binary classification with an artificial labeling and we compared classical and event-based precision and recall metrics.

Regarding model selection, the results show that using event-based F1 score as a metric for model selection can select a model that has better recall at high precision values than models selected by classical F1 score or AUPRG.
However, in other cases the model selected by classical F1 score was better.
The results of model selection also suggest that when using event-based metrics it might be worth trying different sizes of artificial labeling, i.e. try to artificially label either less than or more than $M$\footnote{size of the monitoring window} samples prior to the failure.
In other words, the amount of artificial labeling might be another hyperparameter to tune.

% Calculating the area under event-based precision recall curve can be computationally more expensive than training the model itself and using event-based F1 score, which uses predictions at fixed decision threshold, can lead to selecting a model that overly smooths the predictions.

Regarding decision threshold selection and interpretability, our results show that event-based precision and recall might provide more realistic estimates of the model's precision and recall.
Moreover, using the event-based metrics for decision threshold selection might advise to select a higher decision threshold than when the classical metrics are used.
This implies that according to event-based metrics, better precision can be achieved without much of a loss in recall.
This can be especially useful information when the false alarms are costly and thus precision should be high.

Event-based metrics have several parameters that can be tuned such as the existence weight $\alpha$ or the positional bias function.
It might be interesting to compare how the choice of these parameters affect the selection of the model.
We consider this, however, as out of scope of this thesis.

% To summarize, our experiment shows that event-based metrics are promising metrics for the interpreting the real-world performance of the model failure prediction models.
% Moreover, we identified directions future research.
\section{Experiment --- RUL Prediction of Turbofan Engines}
\label{sec:experiments_rul}

This section describes an experiment where we demonstrate \gls{rul} prediction approach and we compare its evaluation metrics.
\Gls{rul} prediction is a \gls{pdm} approach where the goal is to predict the exact time that remains until a failure occurs --- the \acrfull{rul}.
This approach is suitable in cases when there is a continuous degradation process of the subject.

There exist two main approaches to \gls{rul} prediction: direct RUL prediction and HI-based RUL prediction.
The first approach, the direct \gls{rul} prediction, consists in training a regression model to predict \gls{rul} values retrospectively calculated from run-to-failure data.
The second approach, HI-based RUL prediction, consists in building a time series prediction model that predicts when a \acrfull{hi} of the subject crosses a predefined failure threshold.
The two approaches differ in what data are required for the model to be build (the necessity of run-to-failure data versus the necessity of a single HI and failure threshold) and in what models they use (classical regression vs time series prediction).
In this experiment we choose to demonstrate the direct RUL prediction approach. 

For the purpose of our experiment we choose a data set containing run-to-failure data of turbofan engines \cite{data_set_turbofan} which has already been used in many works regarding RUL prediction \cite{mosallam2016data, ellefsen2019remaining, babu2016deep, peng2019bayesian} and is thus well known in the community of predictive maintenance and related fields.
In contrast with the existing works, where the best performing model is selected based on one specific metric, most commonly RMSE, we evaluate the built models using multiple metrics and we analyze how do the models selected by various metrics differ.

\subsection{Data Set Description}
\label{sec:experiment_rul_data}

\begin{figure}
    \includegraphics[width=\textwidth,keepaspectratio]{%
        experiments_rul_sensor_values.pdf}
	\centering
	\caption{RUL prediction of turbofan engines: Sensor data for one engine. The failure of the engines occurred after the last operation cycle. We can see that a fault has developed somewhere between 50th and 100th cycle and it grows in magnitude until a failure.}
	\label{fig:turbofan_sensors}
\end{figure}

The data set used in this experiment is turbofan engine degradation data set that contains run-to-failure data of hundreds of turbofan engines of the same fleet.
The data of each engine consist of multivariate time series containing measurements from 26 sensors and 3 operational settings.
The time axis is the current number of an engine's operation cycle.

Each engine starts with different degree of initial wear and manufacturing variation which is unknown.
This wear and variation is considered normal, i.e. it is not considered a fault condition.
A fault develops at some point during the engine's operation and grows in magnitude until a failure occurs.
The end of each time series thus marks the engine's failure.
The average length of an engine's operation is approximately 200 cycles.
Figure \ref{fig:turbofan_sensors} shows an example of data of one engine.

Four different training data sets are provided with different operating conditions and different failure modes.
Each training data set consists of the multivariate time series (as described above) for hundreds of engines.
There are four testing sets corresponding to the four training sets, respectively.
Each testing set consists of two files.
The first contains data from tens of engines in the same format as the training data but the data are randomly trimmed from the right side, i.e. the end of the time series does not mark the engines failure.
The second file contains one RUL value for each of the engine in the first file corresponding to the length of the data trimmed from the right, i.e. the RUL at the end of each time series.

The data are provided as CSV files with the time series being in the long format, i.e. each row represents data for one operational cycle of one engine.

\paragraph{Target variable}
The data are run-to-failure with last record being the last measurement before the We failure occurred.
We thus calculate the RUL retrospectively from the data such as for every subject the last record has RUL 1, second from the last record RUL 2 and so on.

\paragraph{Features}
As the features we use the sensor measurements, operational settings and current cycle number.
Seven sensor features and one operational setting feature contain constant values.
Therefore, we remove these features as they do not bring any information.
The sensor measurements contain a lot of noise (see Figure \ref{fig:turbofan_sensors} for illustration).
Therefore we perform smoothing of the sensor values by a rolling mean of size 11 where each sensor value is replaced by the mean value of the past three values (including the current)\footnote{Note, that is is extremely important that the smoothing (or any other kind of rolling operation) must be done by assigning the aggregated value to the most-right element of the window. A centered window for example would use data from the future.}.
At last, we normalize all the features using the values from the training data set.


\begin{figure}[!htb]
	\centering
    \includegraphics[width=.7\textwidth,keepaspectratio]{%
        experiments_rul_design.pdf}
	\caption{RUL prediction of turbofan engines: Design of experiment.}
	\label{fig:experiments_rul_design}
\end{figure}

\subsection{Task Definition}
\label{sec:experiment_rul_task}

The task defined by the authors of the data set is to build a RUL prediction model using the training set and evaluate it using the testing set --- i.e. to predict one RUL value for each engine in the testing set.
However, we consider such evaluation as not being appropriate as a RUL prediction model deployed in production will most probably continuously predict RUL values at each operational cycle.
We think a RUL prediction model should be rather evaluated by RUL predictions from the whole life cycles of the testing subjects rather than by only one randomly selected RUL.
Therefore, we redefine the task as to:
\begin{itemize}
    \item split the original training set (containing the condition monitoring data for each engine's whole life cycle) into new training and testing sets\footnote{splitting should be then made per subject to avoid overfitting};
    \item use the newly created training set for building the model;
    \item use the newly created testing set for the model evaluation.
\end{itemize}
For the purpose of our experiment we use the first of the four provided training sets that contain data from exactly 100 engines that operate under stable conditions and develop a fault in the high pressure compressor.

Moreover, we set a warning window ($W$) so that the predictions close to the failures are excluded from the evaluation, i.e. all predictions made for actual RUL lesser than $W$ are excluded from evaluation.
There are two reasons for using the warning window.
The first is that the predictions of RUL very near to 0 are likely to have high error (e.g. predicted RUL being 2 at actual RUL 1 is equal to error 200 \%).
The second reason is that predictions such close to the failure are commonly of low value in practice as the maintenance action can for example be already be scheduled when predicted RUL is for example 10 (rather than 1 or 2)\footnote{though this of course depends on the specific use case}.
As an average life cycle of the engines is about 200 cycles, we set the warning window to 5 cycles (2.25 \% of the average life cycle).

\subsection{Design of Experiment}
\label{sec:experiment_rul_doe}

The goals of this experiment are to demonstrate \gls{rul} prediction and compare its evaluation metrics.
Commonly, there is a need for the RUL prediction models to have better performance with RUL reaching zero as the maintenance actions are typically performed when the subject's RUL is low (but not yet in the warning window).
Therefore, we design the experiment so that the model selection is done in two steps.
The first step consists in selecting a set of candidate models from a large set of trained models using various evaluation metrics which provide a single score.
The second step then consists in more an analysis of the candidate models in terms of how they perform relatively to RUL, i.e. whether for example some model predicts RUL accurately most of the time (thus having good overall performance) but has poor predictions near the failure.
The individual steps of the experiment design (illustrated in Figure \ref{fig:experiments_rul_design}) are described below.

\subsubsection{Data Splitting}

Split the data into training and testing sets with ratio 4:1.
Perform the split per subject, i.e. all data from one subject are either in the training or the testing set.
We consider the split per subject is crucial so that the model does not learn some subject-specific patterns and so that we do not obtain overly optimistic evaluation results.
The training and testing sets thus contain data about 80 and 20 subjects, respectively.

\subsubsection{Candidate Models Selection}

We use the training set to train the following regression models:
\begin{itemize}
    \item XGBoost (regression version) with n\_estimators $\in [3, 4, 5, 7, 8, 9]$ and max\_depth $\in [16, 32, 64, 128, 256]$;
    \item SVR with gamma $\in [0.01, 0.1, 0.5, 1]$ and C $\in [0.1, 1, 10, 100]$;
\end{itemize}
and evaluate them using 4-fold cross validation\footnote{split the data again per subject, i.e. each testing (training) fold contains 20 (60) subjects}.
We decide to evaluate each model using multiple metrics and later analyze whether they differ in what model they select as the best.

As described in the theoretical part of this thesis in Section \ref{sec:approaches_rul_evaluation}, there exist many metrics and their variants how to evaluate a RUL prediction model.
Therefore, we use only a small representative subset of the described metrics:
\begin{itemize}
    \item root mean squared error (RMSE);
    \item mean absolute percentage error (MAPE);
    \item MAPE@40, i.e. MAPE calculated only on RUL values lower or equal to 40;
    \item MPH$_{10}$, i.e. mean prognostic horizon with $\alpha$ bound 10 --- a mean time prior to a failure so that all the predictions differs from actual RUL by 10 at maximum.
\end{itemize}
We choose RMSE because it is one of the most commonly used metric in scientific articles \cite{mosallam2016data, ellefsen2019remaining, babu2016deep, peng2019bayesian} and it gives the highest weight to the errors at high RUL values.
Next, we choose MAPE as it should give more weight on the errors near the failure.
At last, we choose metrics MAPE@40 and MPH$_{10}$ which should select models only by their performance at low RUL values.

We try all the combinations of the above mentioned hyperparameters of the models totalling in 46 models (30 XGBoost models and 16 SVR models).
We rank the models by mean score across the testing folds for each of the evaluation metrics so that trained model thus obtains 4 ranks all being $\in [1, 47]$.
The lower the rank the better the model by the corresponding metric.

To compare how the evaluation metrics rank the models we visualize a pair plot of the rankings of the models.
As the set of candidate models we then select the models that are ranked as the best by at least one of the metrics.

\subsubsection{Candidate Models Comparison}

We use the testing set to calculate MAPE relative to RUL for every candidate model, i.e. MAPE at RUL 50 is calculated from all the predictions made for samples with an actual RUL 50.
We plot the errors, visually compare whether and how the candidate models differ and discuss which model may be more suitable for the given task and why.

\begin{figure}
	\centering
    \includegraphics[width=\textwidth,keepaspectratio]{%
        experiments_rul_pairplot.pdf}
	\caption{RUL prediction of turbofan engines: Pair plot of the ranks of the tested models.}
	\label{fig:experiments_rul_pairplot}
\end{figure}

\subsection{Results}
\label{sec:experiment_rul_results}

Figure \ref{fig:experiments_rul_pairplot} shows a pair plot of the ranks of the 46 trained models by the evaluation metrics.
We can see that only pair of metrics that roughly agrees on the ranking is MAPE@40 and MPH$_{10}$, although MPH$_{10}$ ranks several SVR models very high and MAPE@40 ranks them low.
RMSE, MAPE and MPH agree on the best model, but highly disagree at the lower ranked models.
MAPE@40, on the other hand, completely disagrees with RMSE and MAPE.

\begin{table}
\centering
\begin{tabular}{lll}
\toprule
{} & \multicolumn{2}{c}{candidate model} \\
{} &                                     A &                                     B \\
\midrule
rank by RMSE    &                            24 &                         1 \\
rank by MAPE    &                            14 &                         1 \\
rank by MAPE@40 &                             1 &                        20 \\
rank by MPH\_10  &                             3 &                         1 \\
RMSE            &                            42.9 &                      37.6 \\
MAPE            &                              30.3 &                      27.5 \\
MAPE@40         &                            38.4 &                      42.3 \\
MPH\_10          &                              29.8 &                      32.8 \\
regressor       &                  XGBoost &                       SVR \\
\multirow{2}{}{hyperparameters}          & n\_estimators: 16 &  C: 100 \\
{} &  max\_depth: 7 & gamma: 0.1 \\
\bottomrule
\end{tabular}
\caption{RUL prediction of turbofan engines: the candidate models.}
\label{tab:experiments_rul_params}
\end{table}

Table \ref{tab:experiments_rul_params} shows parameters and evaluation results of the candidate models, i.e. models selected as best by at least one metric.
We can see that there are two candidate models --- the first, model A, selected by MAPE@40 and the second, model B, selected by the other three metrics.
The first model has very low rank by MAPE and RMSE which is most probably caused by having very high errors at the RUL values higher than 40.
From these single scores themselves, however, we can hardly interpret the models' performance as we do not know how it performs with respect to actual RUL values.
Therefore, we perform the further analysis of the two models by plotting the errors relative to RUL.

\begin{figure}
	\centering
    \includegraphics[width=\textwidth,keepaspectratio]{%
        experiments_rul_relative_mape.pdf}
	\caption{RUL prediction of turbofan engines: MAPE over various RUL values on the testing data set.}
	\label{fig:experiments_rul_relative_mape}
\end{figure}

\begin{figure}
    \centering
    \begin{subfigure}{.49\textwidth}
        \includegraphics[width=\textwidth]{%
            experiments_rul_prediction_11.pdf}
    \end{subfigure}
    \begin{subfigure}{.49\textwidth}
        \includegraphics[width=\textwidth]{%
            experiments_rul_prediction_18.pdf}
    \end{subfigure}
    \begin{subfigure}{.49\textwidth}
        \includegraphics[width=\textwidth]{%
            experiments_rul_prediction_33.pdf}
    \end{subfigure}
    \begin{subfigure}{.49\textwidth}
        \includegraphics[width=\textwidth]{%
            experiments_rul_prediction_57.pdf}
    \end{subfigure}
    \caption{RUL prediction of turbofan engines: examples of predictions for multiple subjects}
    \label{fig:experiments_rul_prediction}
\end{figure}

Figure \ref{fig:experiments_rul_relative_mape} shows MAPE metric relative to RUL values calculated using the testing set.
Figure \ref{fig:experiments_rul_prediction} then shows the candidate models' predictions for four random subjects.
We can see that the SVR model (model B, selected by RMSE, MAPE and MPH) has consistently low MAPE at RUL values higher than 40 but the error then significantly rises with the RUL reaching 0.
The XGBoost model (model A, selected by MAPE@40), on the other hand, has higher errors at RULs higher than 40 but is much more accurate at lower values.
However, we see that the standard deviation of the errors of the XGBoost model is significantly higher than of the SVR model.
From the predictions on the four subjects, we can see that the predictions of XGBoost indeed are very unstable over various RUL values.
Therefore, we consider the SVR model to be better, even though it has slightly worse predictions at low RUL values, as we consider the variance of the XGBoost model to be unsuitable for a practical application.

\subsection{Discussion}

In this experiment we demonstrated RUL prediction on a run-to-failure data set of turbofan engines using a direct RUL prediction, i.e. regression-based, approach. 
Our results show that RMSE and MAPE might highly disagree in model ranking.
This suggests RMSE being unsuitable for RUL prediction model selection as the RUL prediction model is typically required to be more precise with RUL reaching lower values.
Moreover, we demonstrated that using a single score to evaluate the whole model does not give necessary insight to evaluate its real-world performance.
Rather, the errors should be plotted against actual RUL values as this might reveal that e.g. the model has very high variance of predictions over actual RUL values.
That might for example reveal that the prediction has high variance over time which might make the model untrustable in practice --- a model that at one cycle predicts RUL 80 cycles and the second cycle predicts RUL 40 cycles (XGBoost model predictions for subject 11 at actual RUL 65 in Figure \ref{fig:experiments_rul_prediction}) is very likely not to be trusted by the users.

% \section{Summary}
% \label{sec:experiments_discussion}

% In our experiments, we demonstrated fault detection, failure prediction and remaining useful life prediction approaches on real-world data sets and we compared their evaluation metrics.

% \paragraph{Fault Detection}
% We demonstrated fault detection on a data set containing point-based faults in Scania trucks and a predefined cost function that assigned costs to FPs and FNs.
% For model selection, we used accuracy, F1 score, AUPRG and AUROC metrics.
% The results suggest that a fault detection model should be selected by either AUROC or AUPRG scores while it does not matter which of the two is used as they rank the models same.
% Moreover, we demonstrated how the predefined cost function for FPs and FNs helps in selecting the optimal decision threshold.
% However, we expressed doubts whether the assessed final precision and recall scores actually reflect how the model will actually perform in practice as there was no time information taken into account such as how often are the trucks inspected (the faults are only point-based).
% Therefore, we can conclude that the fault detection using data with point-based is easy to model and it is easy to assess its performance.

% %TODO

% The failure prediction approach, on the other hand, may provide much more realistic evaluation of the real-world model's performance
% The more realistic evaluation can be done using event-based precision and recall metrics as the existence of a true prediction in the monitoring window is taken into account.
% On the other hand, it might be challenging to select the best performing model from a large amount of trained models.
% Calculating the area under event-based precision recall curve can be computationally more expensive than training the model itself and using event-based F1 score, which uses predictions at fixed decision threshold, can lead to selecting a model that overly smooths the predictions.
% Using a classical F1 score and classical AUPRG (area under precision-recall-gain curve) then leads to very inconsistent ranking of the models.
% Therefore, we consider as a possible solution to pre-select multiple candidate models (possibly based on different metrics) and select the final model using the event-based precision recall curves (thus using the same procedure as we did in the experiment).

% In the last experiment we demonstrated remaining useful life prediction of turbofan engines.
% We demonstrated that RMSE might be unsuitable metric for comparison of RUL prediction models as it might select models that have high percentage error (MAPE).
% Moreover, we demonstrated that the model selection should not be done using a single score (such as RMSE or MAPE values) but the errors should be rather visualized over RUL values.
% Using this visualization we discovered that the model that had very low percentage errors at low RUL values had very high variance of predictions which might cause the model being untrustable in practice.
% Therefore, we conclude that a RUL prediction models shouldn't be evaluated using a single score.
% Similarly as in case of failure prediction, selecting a set of candidate models, preferably using different evaluation metrics, visually inspecting their performance and select the final model with consultation with a domain expert might be the best solution.