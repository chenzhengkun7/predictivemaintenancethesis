\abstractCS{
Prediktivní údržba je strategie plánování údržby, při níž je údržba naplánována pokud subjekt jeví známky závady nebo je pravděpodobné, že brzy dojde k poruše.
Prediktivní údržba snižuje náklady a zabraňuje prostojům ve srovnání s klasickými strategiemi preventivní a reaktivní údržby. 
Prediktivní údržba může být realizována použitím technik umělé inteligence k vytvoření modelu, který zdravotní stav subjektu na základě dat získaných monitorováním jeho stavu.
Existují však různé přístupy k prediktivní údržbě jako detekce závady, predikce poruch a predikce zbývající užitné životnosti, z nichž každý má odlišné požadavky na data a má jiné cíle.
Každý z těchto přístupů využívá jiné techniky umělé inteligence a kvalita modelů vytvořených dle těchto přístupů by měla být hodnocena dle jiných metrik.
Tato diplomová práce poskytuje přehled přístupů k prediktivní údržbě a pomáhá tak odborníkům zvolit vhodný přístup, techniku umělé inteligence a správnou hodnoticí metriku pro jejich problém.
}
